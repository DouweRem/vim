%----------------------------------------------------------------------------------------
%	PACKAGES AND OTHER DOCUMENT CONFIGURATIONS
%----------------------------------------------------------------------------------------

\documentclass[a4paper]{article} % Default font size is 12pt, it can be changed here
\usepackage{geometry} % Required to change the page size to A4
\usepackage{float} % Allows putting an [H] in \begin{figure} to specify the exact location of the figure
\usepackage{amsmath, amssymb}
\usepackage{titlesec}
\usepackage{graphicx} % Required for including pictures
\usepackage{subcaption}
\usepackage{color}
\usepackage[hidelinks]{hyperref}
\usepackage[sorting=none, style=numeric, maxnames=5, backend=biber]{biblatex}
\addbibresource{BESTAND}
\usepackage{enumitem} %allows easy continueing lists.
\usepackage{siunitx} %Here you can write your things in SIunits
\usepackage{listings} %Listings package is for scripts
\usepackage{cprotect}
\usepackage[dutch, english]{babel}
\usepackage{float}

\setlength{\parskip}{0.25cm}
\setlength{\parindent}{0cm}

% MATHEMATICS 
%\renewcommand{\vec}[1] {\ensuremath{ \overrightarrow{ #1 } }}
% \bra \ket \braket and \proj
\newcommand{\bra}[1]{\ensuremath{\langle #1 \vert}}
\newcommand{\ket}[1]{\ensuremath{\vert #1 \rangle}}
\newcommand{\braket}[2]{\ensuremath{\langle #1 \vert #2 \rangle}}
\newcommand{\ketbra}[2]{\ensuremath{\vert #1 \rangle  \langle #2 \vert}}
\newcommand{\proj}[1]{\ensuremath{\vert #1 \rangle \langle #1 \vert}}
\newcommand{\expect}[1]{\ensuremath{\langle #1 \rangle}}
\newcommand{\op}[1]{\mathrm{#1}}
\newcommand{\abs}[1]{\lvert#1\rvert}
\newcommand{\norm}[1]{\lVert#1\rVert}
\newcommand{\diff}[2]{\frac{\mathrm{d} #1}{\mathrm{d} #2}}
\newcommand{\pd}[2]{\frac{\partial #1}{\partial #2}}
\newcommand{\dirac}[1]{\displaystyle{\not}#1}
\newcommand{\up}{\uparrow}
\newcommand{\down}{\downarrow}

%%%% CODE ENVIRONMENT
\definecolor{mygreen}{rgb}{0,0.6,0}
\definecolor{mygray}{rgb}{0.5,0.5,0.5}
\definecolor{mymauve}{rgb}{0.58,0,0.82}

\lstset{ 
  backgroundcolor=\color{white},   % choose the background color; you must add \usepackage{color} or \usepackage{xcolor}; should come as last argument
  basicstyle=\footnotesize,        % the size of the fonts that are used for the code
  breakatwhitespace=false,         % sets if automatic breaks should only happen at whitespace
  breaklines=true,                 % sets automatic line breaking
  captionpos=b,                    % sets the caption-position to bottom
  commentstyle=\color{mygreen},    % comment style
  extendedchars=true,              % lets you use non-ASCII characters; for 8-bits encodings only, does not work with UTF-8
  frame=single,	                   % adds a frame around the code
  keepspaces=true,                 % keeps spaces in text, useful for keeping indentation of code (possibly needs columns=flexible)
  keywordstyle=\color{blue},       % keyword style
  language=Python,                 % the language of the code
  numbers=left,                    % where to put the line-numbers; possible values are (none, left, right)
  numbersep=5pt,                   % how far the line-numbers are from the code
  numberstyle=\tiny\color{mygray}, % the style that is used for the line-numbers
  rulecolor=\color{black},         % if not set, the frame-color may be changed on line-breaks within not-black text (e.g. comments (green here))
  showspaces=false,                % show spaces everywhere adding particular underscores; it overrides 'showstringspaces'
  showstringspaces=false,          % underline spaces within strings only
  showtabs=false,                  % show tabs within strings adding particular underscores
  stepnumber=1,                    % the step between two line-numbers. If it's 1, each line will be numbered
  stringstyle=\color{mymauve},     % string literal style
  tabsize=3,	                   	 % sets default tabsize to 2 spaces
  title=\lstname                   % show the filename of files included with \lstinputlisting; also try caption instead of title
}

\DeclareCiteCommand{\cite}[\mkbibsuperscript]
  {\iffieldundef{prenote}
     {}
     {\BibliographyWarning{Ignoring prenote argument}}%
   \iffieldundef{postnote}
     {}
     {\BibliographyWarning{Ignoring postnote argument}}%
   \bibopenbracket}%
  {\usebibmacro{citeindex}%
   \usebibmacro{cite}}
  {\supercitedelim}
  {\bibclosebracket}
 
\DeclareMathOperator{\sinc}{sinc}
\begin{document}

%----------------------------------------------------------------------------------------
%	TITLE PAGE
%----------------------------------------------------------------------------------------

\begin{titlepage}

\newcommand{\HRule}{\rule{\linewidth}{0.5mm}} % Defines a new command for the horizontal lines, change thickness here

\center % Center everything on the page
\begin{figure}[H] \center{\includegraphics[width=0.2\linewidth]{/Users/douweremmelts/Documents/Template/logo.png}} \end{figure}
\textsc{\LARGE Universiteit Leiden}\\[1.5cm] % Name of your university/college
\textsc{\Large Physics Experiments}\\[0.5cm] % Major heading such as course name
%\textsc{\large Eindproject}\\[0.5cm] % Minor heading such as course title

\HRule \\[0.4cm]
{ \huge \bfseries Titel}\\[0.4cm] % Title of your document
\HRule \\[1.5cm]

\begin{minipage}{0.45\textwidth}
\begin{flushleft} \large
\emph{Authors:}\\
Douwe \textsc{Remmelts} \small{(\texttt{s2586592})}\\
\end{flushleft}
\end{minipage}
~
\begin{minipage}{0.4\textwidth}
\begin{flushright} \large
\emph{Lecturers:} \\
Dr. ir. Paul \textsc{Logman}\\ % Supervisor's Name
Evert \textsc{Stolte}\\ % Supervisor's Name
\hspace{1cm} \\
\end{flushright}
\end{minipage}\\[4cm]

{\large \today\\Leiden}\\[3cm] % Date, change the \today to a set date if you want to be precise

\vfill % Fill the rest of the page with whitespace

\end{titlepage}

%----------------------------------------------------------------------------------------
%	Abstract
%----------------------------------------------------------------------------------------
\selectlanguage{english}
\begin{abstract}
Lorem ipsum dolor sit amet, consectetur adipiscing elit. Proin convallis, mauris pharetra tempus congue, orci nisl finibus lacus, quis pharetra metus nulla sit amet sem. Mauris facilisis velit scelerisque neque ultrices finibus. Phasellus dictum elementum ex, quis porta tortor rutrum vel. Nulla dignissim justo vel scelerisque scelerisque. Integer quis neque enim. Nullam auctor enim eget convallis blandit. Sed iaculis nulla sapien, vel finibus velit sagittis a. Proin tincidunt nec metus vel laoreet. Aliquam porttitor finibus mauris vel volutpat. Phasellus vulputate eros ipsum, eget vehicula sem feugiat vitae. In molestie tellus ac arcu rutrum accumsan.
\end{abstract}
%----------------------------------------------------------------------------------------
%	TABLE OF CONTENTS
%----------------------------------------------------------------------------------------
\setcounter{tocdepth}{2}
\tableofcontents % Include a table of contents

\newpage % Begins the essay on a new page instead of on the same page as the table of contents

%----------------------------------------------------------------------------------------
%	INTRODUCTION
%----------------------------------------------------------------------------------------

%----------------------------------------------------------------------------------------
%	THEORY
%----------------------------------------------------------------------------------------

%----------------------------------------------------------------------------------------
%	BIBLIOGRAPHY
%----------------------------------------------------------------------------------------
\newpage
\printbibliography

%----------------------------------------------------------------------------------------
%	APPENDIX: THE CODE
%----------------------------------------------------------------------------------------

\newpage

\end{document}


